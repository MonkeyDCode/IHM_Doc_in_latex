\documentclass[12pt,letterpaper]{article}
\usepackage[latin1]{inputenc}
\usepackage{amsmath}
\usepackage{amsfonts}
\usepackage{amssymb}
\usepackage{graphicx}
\usepackage[left=4cm,right=3cm,top=4cm,bottom=3cm]{geometry}
\author{Por: Mario Alberto Guti�rrez Corral}
\title{Ingenier�a de la usabilidad}
\date{25 de Junio de 2015}
%comentario

\begin{document}
\begin{titlepage}
\vspace*{0.15in}
\begin{center}
\vspace*{-1in}
\begin{figure}[htb]
\begin{center}
\includegraphics[width=8cm]{./logoUAEM}
\end{center}
\end{figure}
\vspace*{0.15in}
FACULTAD DE INGENIER�A\\
\vspace*{0.15in}
INGENIER�A EN COMPUTACI�N \\
\vspace*{0.6in}
\begin{large}
INTERACCI�N HOMBRE M�QUINA\\
\end{large}
\vspace*{1.5in}
\begin{Large}
\textbf{PROCEDIMIENTO PARA PRUEBA DE SOFTWARE} \\
\end{Large}
\vspace*{1in}
\begin{large}
POR: MARIO ALBERTO GUTI�RREZ CORRAL\\ 
\hspace{1cm}ALVARO TONATIHU FABIAN SILVERIO \\
\hspace{2cm}JOS� MIGUEL ALONSO MU�OZ\\
\end{large}
\vspace*{0.6in}
\rule{80mm}{0.1mm}\\
\vspace*{0.1in}
\begin{large}
Profesor: \\
Dr. Marcelo Romero Huertas \\
\end{large}
\end{center}

\end{titlepage}
%\maketitle
\newpage

\tableofcontents
\newpage

\section{Introducci�n}
\bigskip
Este articulo presenta los procedimientos para realizar la prueba de aplicaciones del software "smile" y define los t�rminos asociados con la misma.\newline
Una de las metas de la Ingenier�a  de Software es aumentar el nivel de correcci�n del software. El prop�sito de la prueba es dar una medida de la correcci�n de  programa cuando �ste se est� desarrollando. \newline
La prueba es parte integral del ciclo de dise�o y por tanto debe verificarse la correcci�n del programa cuando �ste se est� desarrollando.\newline

\section{Prueba de Funcionalidad}
\subsection{Objetivo}
Verificar la funci�n del sistema al fijar la tensi�n en la validaci�n de las funciones, m�todos, servicios y casos de uso.
\bigskip
\subsection{Metas}
Validar que la aplicaci�n: \newline
\begin{itemize}
    \item Cumpla con los requisitos funcionales especificados en el dise�o de la soluci�n.
    \item Cumpla con los requisitos no funcionales especificados en el dise�o de la soluci�n.
    \item Cumpla con las restricciones de entrada y salida de la informaci�n de cada caso de uso.
    \item Cumpla integralmente con la estructura referencial especificada en el Mapa de Navegaci�n
\end{itemize}

\section{Prueba de}
\subsection{Objetivo}

\subsection{Metas}


\section{Prueba de}
\subsection{Objetivo}

\subsection{Metas}


\section{Prueba de}
\subsection{Objetivo}

\subsection{Metas}


\section{Prueba de}
\subsection{Objetivo}

\subsection{Metas}


\section{Prueba de}
\subsection{Objetivo}

\subsection{Metas}



\end{document}