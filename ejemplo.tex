\documentclass[12pt,letterpaper]{article}
\usepackage[latin1]{inputenc}
\usepackage{amsmath}
\usepackage{amsfonts}
\usepackage{amssymb}
\usepackage{graphicx}
\usepackage[left=4cm,right=3cm,top=4cm,bottom=3cm]{geometry}
\author{Por: Mario Alberto Guti�rrez Corral}
\title{Ingenier�a de la usabilidad}
\date{25 de Junio de 2015}
%comentario

\begin{document}
\maketitle
\newpage
%\centerline{\underline{\Huge\textbf{La Usabilidad}}}
\section{Introducci�n}
\subsection{�Por qu� "Usabilidad"?}
"La capacidad de un software de ser comprendido, aprendido, usado y ser atractivo para el usuario, en condiciones espec�ficas de uso" [ISO 9126]
\newline
"El grado en que un producto puede ser utilizado por usuarios espec�ficos para conseguir objetivos espec�ficos con efectividad, eficiencia y satisfacci�n en un determinado contexto de uso" [ISO 9241]
\newline
Hoy en d�a el software se encuentra en casi todos los campos de la actividad humana. Todos somos usuarios de sistemas inform�ticos.
\newline
Tomando en cuenta  est�s dos referencias anteriores podemos decir que se espera que los productos de software satisfagan ciertas normas y est�ndares de calidad.
	
\section{Ingenier�a de la Usabilidad}

La ingenier�a depende de la interpretaci�n en un contexto compartido de significados, objetivos acordados y un conocimiento de c�mo se va a juzgar una terminaci�n satisfactoria. El �nfasis en la ingenier�a de la usabilidad es en saber exactamente qu� criterio va a ser usado para juzgar la usabilidad de un producto.

El test final de la usabilidad de un producto es basado en las mediciones de la experiencia de los usuarios con �l.

En relaci�n con el ciclo de vida del software, una de las caracter�sticas importantes de la ingenier�a de la usabilidad es la inclusi�n de una especificaci�n de usabilidad, formando parte de la especificaci�n de requerimientos, que se concentra en las caracter�sticas de la interacci�n usuario-sistema que contribuye a la usabilidad del producto.



\end{document}